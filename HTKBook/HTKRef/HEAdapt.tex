%/* ----------------------------------------------------------- */
%/*                                                             */
%/*                          ___                                */
%/*                       |_| | |_/   SPEECH                    */
%/*                       | | | | \   RECOGNITION               */
%/*                       =========   SOFTWARE                  */ 
%/*                                                             */
%/*                                                             */
%/* ----------------------------------------------------------- */
%/*         Copyright: Microsoft Corporation                    */
%/*          1995-2000 Redmond, Washington USA                  */
%/*                    http://www.microsoft.com                */
%/*                                                             */
%/*   Use of this software is governed by a License Agreement   */
%/*    ** See the file License for the Conditions of Use  **    */
%/*    **     This banner notice must not be removed      **    */
%/*                                                             */
%/* ----------------------------------------------------------- */

\newpage
\mysect{HEAdapt}{HEAdapt}

\subsection{Function}
\index{headapt@\htool{HEAdapt}|(}
This program is used to perform 
adaptation of a set of HMMs using either maximum likelihood linear 
regression (MLLR), maximum a-posteriori (MAP) or both. 
The default is MLLR.
In order to perform the adaptation, the first stage requires a
state/frame alignment. As such the initial operation of \htool{HEAdapt}
follows \htool{HERest} closely. The adaptation training
data consists of one or more utterances each of which has a 
transcription in the form of a standard label file (segment
boundaries are ignored).  For each training utterance, a
composite model is effectively synthesised by concatenating
the phoneme models given by the transcription.  Each mixture
component's accumulators in \htool{HEAdapt}
are updated simultaneously by performing a standard Baum-Welch pass over
each training utterance using the composite model.
\htool{HEAdapt} will also prune the
$\alpha$ and $\beta$ matrices, just as \htool{HERest}.  
  
\htool{HEAdapt} is intended to operate on HMMs which have been fully
trained using \htool{HCompV}, \htool{HInit}, \htool{HRest}, \htool{HERest}.
\htool{HEAdapt} supports multiple mixture diagonal covariance Gaussian
HMMs only (i.e. \texttt{PLAINHS} and \texttt{SHAREDHS} systems),
with a single data stream only, and parameter tying within and between models. 
\htool{HEAdapt} also supports tee-models
(see section~\ref{s:teemods}), for handling optional silence and non-speech
sounds. These may be placed between the units (typically words or phones)
listed in the transcriptions, but they cannot be used at the start or end of a
transcription. Furthermore, chains of tee-models are not permitted.

After accumulating statistics, \htool{HEAdapt} estimates the mean and
(optionally) the variance transforms.  \htool{HEAdapt} will output
either the adapted HMM set (as an MMF), or a transform model file
(TMF). The TMF can then be applied to the original model set (for
instance when using \htool{HVite}). Note that with MAP adaptation a
transform is not available and a full HMM set must be output.

When \htool{HEAdapt} is being run to calculate multiple regression
transforms, the model set being adapted must contain a regression
class tree. The regression class tree is constructed using the {\tt
RC} edit command in \htool{HHEd}.

\subsection{Use}{}

\htool{HEAdapt} is invoked via the command line
\begin{verbatim}
   HEAdapt [options] hmmList adaptFile ...
\end{verbatim}
This causes the set of HMMs given in {\tt hmmList} to be loaded.
The given list of
adaptation training files is then used to perform one adaptation
cycle. As always, the list of training files can be stored in a script
file  if required.  On completion, \htool{HEAdapt} outputs new updated 
versions of each HMM definition or a new transform models file.

The detailed operation of \htool{HEAdapt} is controlled by the following
command line options
\begin{optlist}

  \ttitem{-b N} Set the number of blocks to be used in the block
      diagonal matrix representation of the mean transformation. This
      option will override the config setting \texttt{HADAPT:BLOCKS},
      or if this is not set the default number of blocks is 1.

  \ttitem{-c f} Set the minimum forward probability fixed distance for
      the alpha pruning to {\tt f}. Restrict the computation of the
      $\alpha$ values to just those for which the total log likelihood
      $\alpha_j(t)\beta_j(t)$ is within distance {\tt f} of the total
      likelihood (default 10.0).
 
  \ttitem{-d dir} 
      Normally \htool{HEAdapt} looks for HMM definitions
      (not already loaded via MMF files) 
      in the current directory.  This option tells \htool{HEAdapt} to look in
      the directory {\tt dir} to find them.

  \ttitem{-f field desc} Set the description field {\tt field} in the
      transform model file to {\tt desc}. Currently the choices for
      {\tt field} are {\tt uid}, {\tt uname}, {\tt chan} and {\tt desc}.

  \ttitem{-g} Perform global adaptation only. 

  \ttitem{-i N} Update the transforms (incrementally) after
      accumulating statistics every {\tt N} utterances. The default
      operation is static adaptation, i.e. after seeing ALL the
      adaptation data.

  \ttitem{-j f} MAP adaptation with scaling factor {\tt f}. The
      default operation is MLLR adaptation. If MAP adaptation is to be
      performed the default value of {\tt f} is 15.0
 

  \ttitem{-k} Use MLLR to transform the HMM model set before performing MAP

  \ttitem{-m f}  Set the minimum threshold occupation count for a
      regression class to {\tt f}. A separate regression
      transformation will be generated at the lowest level in the tree
      for which there is sufficient occupancy (data). This option will
      override the config setting \texttt{HADAPT:OCCTHRESH}. The
      default setting is 700.0.
  
  \ttitem{-o ext}  This causes the file name extensions of the
      original models (if any) to be replaced by {\tt ext}.

  \ttitem{-t f [i l]} Set the pruning threshold to {\tt f}.  During the 
      backward probability calculation, at
      each time $t$ 
      all (log) $\beta$ values falling more than {\tt f} below the
      maximum $\beta$ value at that time are ignored.  During the
      subsequent forward pass, (log) $\alpha$ values are only
      calculated if there are corresponding valid $\beta$ values.
      Furthermore, if the ratio of the $ \alpha \beta $ product divided
      by the total probability (as computed on the backward pass)
      falls below a fixed threshold then those values of $\alpha$
      and $\beta$ are ignored. Setting {\tt f} to zero disables
      pruning  (default value 0.0).  Tight pruning thresholds can
      result in \htool{HEAdapt} failing to process an utterance.
      if the {\tt i} and {\tt l} options are given, then a pruning
      error results in the threshold being increased by {\tt i} and
      utterance processing restarts.  If errors continue, this procedure will 
      be repeated until the limit {\tt l} is reached.
      
  \ttitem{-u flags} By default \htool{HEAdapt} creates transforms for
      the means only. This option causes the parameters indicated by
      the {\tt flags} to be created; this argument is a string
      containing one or more of the letters {\tt m} (mean) and {\tt v}
      (variance). The presence of a letter
      enables the creation of the corresponding part of the transform.

  \ttitem{-w f}  This sets the minimum variance (i.e. diagonal element of
      the covariance matrix) to the real value {\tt f} (default value
      0.0).
      
  \ttitem{-x ext}  By default, \htool{HEAdapt} expects a HMM definition for 
      the label X to be stored in a file called {\tt X}.  This
      option causes \htool{HEAdapt} to look for the HMM definition in the
      file {\tt X.ext}.
  \ttitem{-B} Output the HMM definition files in binary format. If
      outputting a tmf, then this option specifies binary output for
      the tmf.

\stdoptF
\stdoptG
\stdoptH
\stdoptI

 \ttitem{-J tmf} Load a transform set from the transform model file
      {\tt tmf}. The {\tt tmf} is used to transform the {\tt mmf}
      before performing the state/frame alignment, and a transform is
      calculated based on this state/frame alignment and the {\tt mmf}.
      
  \ttitem{-K tmf} Save the transform set in the transform model file 
      {\tt tmf}.

\stdoptL
\stdoptM
\stdoptX

\end{optlist}
\stdopts{HEAdapt}

\subsection{Tracing}

\htool{HEAdapt} supports the following trace options where each
trace flag is given using an octal base
\begin{optlist}
   \ttitem{00001} basic progress reporting.
   \ttitem{00002} show the logical/physical HMM map.
\end{optlist}

Trace flags are set using the \texttt{-T} option or the  \texttt{TRACE} 
configuration variable.

The library that \htool{HEAdapt} utilises, called \htool{HAdapt} supports other
useful trace options. For library modules, tracing has to be performed via 
the config file and the module name must prefix the trace (e,g \texttt{HADAPT:TRACE=0001}). 
The following are \htool{HAdapt} trace options
where each trace flag is given using an octal base
\begin{optlist}
   \ttitem{00001} basic progress reporting.
   \ttitem{00002} trace on the accumulations.
   \ttitem{00004} trace on the transformations.
   \ttitem{00010} output the auxiliary function score.
   \ttitem{00020} regression classes input/output tracing.
   \ttitem{00040} regression class tree usage.
   \ttitem{00200} detailed trace for accumulations at the class level.
\end{optlist}
\index{headapt@\htool{HEAdapt}|)}


%%% Local Variables: 
%%% mode: latex
%%% TeX-master: "../htkbook"
%%% End: 
