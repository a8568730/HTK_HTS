%/* ----------------------------------------------------------- */
%/*                                                             */
%/*                          ___                                */
%/*                       |_| | |_/   SPEECH                    */
%/*                       | | | | \   RECOGNITION               */
%/*                       =========   SOFTWARE                  */ 
%/*                                                             */
%/*                                                             */
%/* ----------------------------------------------------------- */
%/*         Copyright: Microsoft Corporation                    */
%/*          1995-2000 Redmond, Washington USA                  */
%/*                    http://www.microsoft.com                 */
%/*                                                             */
%/*   Use of this software is governed by a License Agreement   */
%/*    ** See the file License for the Conditions of Use  **    */
%/*    **     This banner notice must not be removed      **    */
%/*                                                             */
%/* ----------------------------------------------------------- */
%
% General HTK macros - Julian Odell 23/04/97
%             (made plain version) Rich Wareham 19/07/00
%

% HTK Documentation Utilities

\usepackage[dvips]{epsfig}
\usepackage{html}
\usepackage{color}

%begin{latexonly}
\usepackage{calc}
\usepackage{ifthen} % Only include for tex
\newcommand{\iftex}[1]{ #1 }    % Do tex
\newcommand{\ifhtml}[1]{    }   % Dont do HTML
%end{latexonly}
\html{
 \newcommand{\iftex}[1]{    }   % Dont do tex
 \newcommand{\ifhtml}[1]{ #1 }  % Do HTML

 % And set the page colour
 \pagecolor{white}
}

\newcommand{\ifoddpage}[3] { 
  \ifthenelse{\isodd{\pageref{f:#1}} } {#2} {#3} 
}
\newcommand{\ifevenpage}[3] { 
  \ifthenelse{\isodd{\pageref{f:#1}} } {#3} {#2} 
}

\newcommand{\htmlparbox}[2] {
  \ifhtml{ \parbox{#1} {#2} }
}

\newcommand{\figDir}{HTKFigs/}

% Global Constants and document settings

%begin{latexonly}
\setlength{\oddsidemargin}{\oddsidemargin - 1cm}
\setlength{\textwidth}{\textwidth + 3cm}
\setlength{\topmargin}{\topmargin - 2cm}
\setlength{\textheight}{\textheight + 4cm}
%end{latexonly}\newlength{\pictextwidth}


% Basics

%\href{key}

% Hack to allow us to generate PDF and PS. 
%begin{latexonly}
  \providecommand{\href}[1]{\ref{#1}}
  \renewcommand{\href}[1]{\ref{#1}}
%end{latexonly}

% Only do thin if we are generating HTML
\html{
  \newcommand{\href}[1]{\ref{#}}
}

%\hequation{equation}{eqlabel}
\newcommand{\hequation}[2]{
\begin{equation} #1 \label{e:#2}
\end{equation}
}

% Headings

\newcounter{tabctr}[chapter]
\newcounter{figctr}[chapter]

\def\thetabctr{\thechapter.\arabic{tabctr}}
\def\thefigctr{\thechapter.\arabic{figctr}}
\def\thesecctr{\thechapter.\arabic{section}}

\newcommand{\mychaptitle}{}
\newcommand{\mysecttitle}{}

\newcommand{\mychap}[2]{
  \newpage
  \chapter{#1}\label{c:#2}
  \iftex{ \renewcommand{\mychaptitle}{\bfseries \rmfamily #1}
          \markboth{\mychaptitle}{\mysecttitle} }
}

\newcommand{\mysect}[2]{
  \section{#1}\label{s:#2}
  \iftex{ \renewcommand{\mysecttitle}
               { \bfseries \rmfamily \thesecctr \hspace{0.5cm} #1}
          \markboth{\mychaptitle}{\mysecttitle} }
}

\newcommand{\mysubsect}[2]{
  \subsection{#1}\label{s:#2}
%  \iftex{ \markboth{\mychaptitle}{\mysecttitle\ - #1} } % Too long
}

% Formatting Macros

\newcommand{\htool}[1]{\textsc{#1}}
\newcommand{\ttitem}[1]{\item[{\tt #1}]}
\newcommand{\fwitem}[2]{\item[{\makebox[#1][l]{\tt #2}}]}
\newcommand{\mathitem}[1]{\item[$ #1 $]}
\newcommand{\incop}{\mbox{$ \:+\!\!=\; $}}
\newcommand{\bm}[1]{{\mbox{\boldmath $#1$}}}

\newenvironment{optlist}{
  \begin{list}{??}{
  }
}{
  \end{list}
}

\newenvironment{varlist}{
  \begin{list}{??}{
  }
}{
  \end{list}
}

\newenvironment{program}{
  \ttfamily
  \begin{tabbing}
    +++ \= ++ \= ++ \= ++ \= ++ \= ++ \= ++ \= ++ \= ++ \= ++ \= ++ \= ++ \= \kill
}{
  \end{tabbing}
}

% Table captions are quite special

%\tabcap{tabname}{caption}
\newcommand{\tabcap}[2]{
  \refstepcounter{tabctr}
  \label{t:#1}
  \mbox{ \rule{0cm}{0.6cm} 
           \textbf{ Table. \thetabctr\ \ #2} }
}

% All the various figure macros

%\dofig{picname}{width_in_mm}{caption}{capspace}
\newcommand{\dofig}[4]{
    
   \begin{center}\parbox{#2mm} { \begin{center}
      \setlength{\epsfxsize}{#2mm}
      \epsfbox{\figDir/#1.eps}  %  DIRECTORY OPTION
      \refstepcounter{figctr}
      \begin{center}
        \textbf{ Fig. \thefigctr\ \  #3}   
      \end{center}     \label{f:#1} \end{center} } \end{center}

}

%\putfig{picname}{width_in_mm}{caption}
\newcommand{\putfig}[3]{
   \dofig{#1}{#2}{#3}{0}
}

%\centrefig{picname}{width_in_mm}{caption}
\newcommand{\centrefig}[3]{
   \dofig{#1}{#2}{#3}{0}
}

%\sidefig{picname}{width_in_mm}{caption}{capspace}{text}
\newcommand{\sidefig}[5]{
  \dofig{#1}{#2}{#3}{#4}
  #5
}

%\sidepic{picname}{width_in_mm}{text}
\newcommand{\sidepic}[3]{
  
  \begin{center}\parbox{#2mm} { \begin{center}
    \setlength{\epsfxsize}{#2mm}
    \epsfbox{\figDir/#1.eps}  %  DIRECTORY OPTION 
    \label{f:#1} \end{center} }\end{center}

  {#3}
}

% Now programs

%\doprog{picname}{width_in_mm}{caption}{program}
\newcommand{\doprog}[4]{
    
    \begin{center} \parbox{#2mm} {\noindent
    \fbox{ \parbox{#2mm}{ \hspace{#2mm}
       \begin{program}#4\end{program}
    } }

    \refstepcounter{figctr}
    \label{f:#1}

    \begin{center}
      \textbf{ Fig. \thefigctr\ \ #3}   
    \end{center}   
    } \end{center} 

}

%\sideprog{progname}{width_in_mm}{caption}{prog}{text}
\newcommand{\sideprog}[5]{
  \doprog{#1}{#2}{#3}{#4} 
   #5
}

%\progbyprog{width_in_mm}{progname1}{progname2}{caption1}{caption2}{prog1}{prog2}
\newcommand{\progbyprog}[7]{
  \doprog{#2}{#1}{#4}{#6}
  \doprog{#3}{#1}{#5}{#7}
  \vspace{0.3cm}
}


%\putprog{progname}{width_in_mm}{caption}{prog}
\newcommand{\putprog}[4]{
  
  \begin{center} \begin{figure}
    \vspace{5mm}
    \begin{center}
      \framebox[#2mm]{
      \begin{minipage}{#2mm}
           \begin{program}#4\end{program}
       \end{minipage} }

       \refstepcounter{figctr}
       \label{f:#1}
       \vspace{0.4cm}
         \mbox{ \textbf{ Fig. \thefigctr\ \ #3} }
    \end{center}
  \end{figure} \end{center}

}

% This section includes commands for each document covered
% 
% Currently this is
%  The HTKBook
%  The HAPIBook
%  The grapHvite book
%  Inside HTK
%  Inside grapHvite
%

% HTKBook stuff

\newcommand{\HTK}{HTK}
\newcommand{\version}{3.0}
\newcommand{\inthisversion}{In \HTK\ \version ,\ }
\newcommand{\HTKFF}{\HTK}
\newcommand{\HTKBook}{\textit{HTK Book}}
\newcommand{\ESPSwaves}{\textit{ESPS/waves+}}
\newcommand{\erno}[1]{\fwitem{2.0cm} {\ \ \ $#1$}}
\newcommand{\module}[1]{\item[#1]{\ ~}}
\newcommand{\refpart}{Reference Section}

%\hmkw{hmmKeyword}
\newcommand{\hmkw}[1]{\mbox{$<$\textsf{#1}$>$}}

%\hmmc{hmmMacro}
\newcommand{\hmmc}[2]{\mbox{$\sim$\textsf{#1 ``#2''}}}

%\hmmt{hmmMacro}
\newcommand{\hmmt}[1]{\mbox{$\sim$\textsf{#1}}}


% HAPIBook stuff

\newcommand{\HAPI}{{HAPI}}
\newcommand{\NULL}{{\tt NULL}}
\newcommand{\findex}[1]{\index{#1@{\bf #1}}}
\newcommand{\pindex}[1]{\index{configuration parameters!#1}}
\newcommand{\java}{{\sc Java}}
\newcommand{\jhapi}{{\sc Jhapi}}

% Common grapHvite phrases

\newcommand{\grapHvite}{\textbf{\textit {grapHvite}}}
\newcommand{\GNB}{\grapHvite\ Netbuilder}
\newcommand{\grapHviteOverview} {{\it \grapHvite\ Overview}}
\newcommand{\seeSec}[1]{see section {\bf \ref{s:#1}}}
\newcommand{\seeSub}[1]{see {section \bf \ref{s:#1}}}
\newcommand{\seeFig}[1]{see Fig {\bf \href{f:#1}}}
\newcommand{\fig}   [1]{figure {\href{f:#1}}}
\newcommand{\seeApp}[1]{see Appendix {\bf \ref{c:#1}}}
\newcommand{\app}[1]{Appendix {\bf \ref{c:#1}}}
\newcommand{\sub}[1]{section {\bf \ref{s:#1}}}
\newcommand{\sect}[1]{section {\bf \ref{s:#1}}}

% GrapHvite Specific Formats

\newcommand{\facility}[1]{{\it #1}}
\newcommand{\key}[1]{{\sl #1}}
\newcommand{\ul}[1]{\underline{#1}}
\newcommand{\gTerm}[1]{{\bf #1}}
\newcommand{\network}[1]{{\it #1}}
\newcommand{\config}[1]{{\it #1}}
\newcommand{\bb}[1]{{\bf #1}}

% Colourboxes need some highlighting for HTML

 \newcommand{\button}[1]{\colorbox[gray]{0.83}{{\small{\textsf{#1}}}}}
 \newcommand{\menu}[1]{\colorbox[gray]{0.83}{{\small \textbf{#1}}}}
 \newcommand{\menuItem}[1]{\colorbox[gray]{0.83}{{\small{ #1}}}}
 \newcommand{\gLabel}[1]{{\colorbox[gray] {0.83} {\small{\textsf {#1}}}}}  

\newcommand{\moptitem}[1] { \optitem{\menuItem{#1}}}
\newenvironment{mylist}[1]{
  \begin{list}{}{
     \settowidth{\labelwidth} {#1} 
     \setlength{\leftmargin}{\labelwidth}
     \addtolength{\leftmargin}{\labelsep}
     \setlength{\parsep}{0.5ex plus0.2ex minus 0.2ex} 
     \setlength{\itemsep}{0.3ex}
     \renewcommand{\makelabel}[1]{##1\hfill}
} } {
 \end{list}
}

\newenvironment{widelist}{
   \begin{list}{??}{
       \setlength{\itemindent}{0cm}
       \setlength{\rightmargin}{0.5cm}
       \setlength{\leftmargin}{2.5cm}
       \setlength{\listparindent}{0cm}
       \setlength{\labelwidth}{\textwidth}
       \setlength{\labelsep}{0.5cm} }
}{
   \end{list}
}

\newcommand{\doMenu}[1]{
  \setlength{\epsfysize} {5cm}
  \raisebox{-5cm}
  {\epsfbox{#1menu.eps}}  %  DIRECTORY OPTION
}

\newcommand{\mysubsub}[2]{
   \subsubsection{#1}\label{s:#2}
}


%%% Local Variables: 
%%% mode: latex
%%% TeX-master: "htkbook"
%%% End: 
